% !TEX TS-program = xelatex
% !TEX encoding = UTF-8 Unicode
% !Mode:: "TeX:UTF-8"

\documentclass{resume}
\usepackage{zh_CN-Adobefonts_external} % Simplified Chinese Support using external fonts (./fonts/zh_CN-Adobe/)
%\usepackage{zh_CN-Adobefonts_internal} % Simplified Chinese Support using system fonts
\usepackage{linespacing_fix} % disable extra space before next section
\usepackage{cite}

\begin{document}
\pagenumbering{gobble} % suppress displaying page number

\name{贺文宁}

% {E-mail}{mobilephone}{homepage}
% be careful of _ in emaill address
\contactInfo{h919313096@gmail.com}{152-0290-0250}{MyBlog:hewenning.com}{GitHub @hewenning}
% {E-mail}{mobilephone}
% keep the last empty braces!
%\contactInfo{xxx@yuanbin.me}{(+86) 131-221-87xxx}{}

\section{教育背景}
\datedsubsection{\textbf{西安理工大学},电子科学与技术,学士,专业前30\%\textit{}}{2015.9 - 2018.7}



\section{实习和项目经历}
\datedsubsection{\textbf{松山湖国际机器人研究院-Software Engineer Intern|}技术栈:C++,Qt,Socket}{2017.7-2017.8}
\begin{itemize}
  \item 参考旧版MFC框架,利用Qt框架设计新的AGV无人搬运车的调度软件
  \item 参与整体调度系统的设计,并提出一些实现想法
\end{itemize}


\datedsubsection{\textbf{基于Turtlebot2机器人的人体姿态识别研究|}技术栈:Python,Linux,ROS,TensorFlow,Andriod}{2017.3-2017.12}
\begin{itemize}
  \item 完成机器人的部署和搭建,包括ROS环境和硬件平台
  \item 配置单目相机的驱动,并用此进行图像数据的收集,与人体姿态识别负责人进行对接
\end{itemize}

\datedsubsection{\textbf{Learning-Log|}技术栈:Python,Django,Bootstrap}{2016.11-2016.12}
\begin{itemize}
	\item 利用Django框架中的表单来完成登陆注册的后台逻辑,使用Bootstrap来美化界面
	\item 尽可能地重构代码,并进行单元测试,代码地址:\textit{https://github.com/hewenning/Learning-Logm}
\end{itemize}

\datedsubsection{\textbf{助老平台APP|}技术栈:C++,OpenCV,Android NDK,Bmob}{2017.5-2017.12}
\begin{itemize}
  \item 利用OpenCV设计一个可以提取身份证号信息的算法,并利用Java中的JNI层调用身份证号OCR算法,
  将其设计成一个可供Android端调用的API
  \item 利用Bmob进行后端的部署,完成可以互相接收语音消息的基本功能
\end{itemize}

\datedsubsection{\textbf{可穿戴智能设备|}技术栈:C,Arduino,GSM,Android}{2016.12-2017.5}
\begin{itemize}
	\item 通过Arduino的IO口检测加速度传感器的瞬时异常加速度变化,发送报警短信
	\item 当收到报警短信时,在高德地图上定位出报警点
\end{itemize}




\section{获奖经历}
% increase linespacing [parsep=0.5ex]
\begin{itemize}[parsep=0.2ex]
  \item 大学参与了一些科研训练,有一篇比较水的论文和一篇未发但是参与的关于ELM的论文
  \item Robomasters2016全国大学生机器人大赛全国三等奖。
  \item Robomasters2016全国大学生机器人大赛西部赛区一等奖。
  \item 校科技节一等奖。
  \item 优秀志愿者荣誉若干。
  \item 个人博客\textit{http://www.hewenning.com},更多小玩具见 \textit{https://github.com/hewenning}
\end{itemize}



\section{社区实践}
% increase linespacing [parsep=0.5ex]
\begin{itemize}[parsep=0.2ex]
  \item 乐于参与开源社区活动,喜欢做开源分享,是我们学校的Linux社的发起人,影响了一些学生开始接触开源项目
  \item 大学期间暑期期间去山村支教,至今仍然和那里的一些孩子有联系
  \item 校机器人俱乐部成员、校青年志愿者协会文宣部负责人、校网络新闻工作室负责人
  \item 在知乎写文章和回答问题,到现在已经有1000+关注,3600+赞同和2100+收藏,有自己的写作专栏并收到过写书邀请,
  知乎主页:\textit{https://www.zhihu.com/people/hewenning/activities}
\end{itemize}



\section{IT技能}
% increase linespacing [parsep=0.5ex]
\begin{itemize}[parsep=0.2ex]
	\item \textbf{编程语言}: Python、C++、Java
	\item \textbf{平台}: Linux、Windows、Android
	\item \textbf{数据库}: MySQL、Redis
	\item \textbf{其他}:MCU、Arduino、ROS
\end{itemize}



\section{个人总结}
本人是电子科学与技术专业,在校成绩优秀,具有扎实的半导体、集成电路和计算机基础知识;对算法有极大的热情和兴趣,平时会总结自己写些文字,喜欢接受挑战,研究新的技术,自我驱动力强

\end{document}
